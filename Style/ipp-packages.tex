\documentclass[12pt]{report}
\usepackage[T1]{fontenc}
\usepackage[frenchb]{babel}             % Chargement de la mise en page fran�aise
\usepackage{geometry}                   % Pour param�trer la mise en page (marges, etc.)
\usepackage{setspace}                   % Pour modifier l'espacement des lignes
\usepackage{color}                      % Pour utiliser la couleur dans le texte
\usepackage{fancyhdr}                   % Pour personnaliser les en-t�tes
\usepackage{pdfpages}                   % Pour inclures des pages pdf au moment de la compilation
\usepackage{sectsty}                    % Pour changer le style des titres de chapitres, de sections, etc.
\usepackage{charter}                    % Pour charger la fonte "Charter BT Roman"
\usepackage{natbib}                     % Pour personnaliser le style de la biblio
\usepackage{amsmath,amssymb, amsthm}    % N�cessaire pour les formules
\usepackage{eurosym}                    % Pour le symbole "euro"
\usepackage{booktabs}                   % Pour utiliser des lignes de tableaux de diff�rentes �paisseurs
\usepackage{enumitem}                   % Pour personnaliser le style des listes � puces
\usepackage{array}
\usepackage{arcs}                       % Pour r�aliser le logo TAXIPP
\usepackage{slashbox}                   % Pour r�aliser le logo TAXIPP
\usepackage{MnSymbol}                   % Pour r�aliser le logo TAXIPP
\PassOptionsToPackage{usenames,dvipsnames,svgnames}{xcolor} % Pour r�aliser le logo TAXIPP
\usepackage{xcolor}                     % Pour r�aliser le logo TAXIPP
\usepackage{longtable}                  % Pour r�aliser des tableaux sur plusieurs pages
\usepackage[pdftitle={Guide m�thodologique IPP -- TAXIPP 0.0},
            pdfauthor={Institut des politiques publiques},
            colorlinks=true,linkcolor=black,citecolor=black,urlcolor=blue]{hyperref}  %pour que le fichier PDF contiennent des liens hypertexte

% Taille des marges %
% Version impression
%\geometry{a4paper, vmargin=2.5cm, hmargin=2.5cm, top=1.3in, headheight=15pt, bindingoffset=0.8cm, twoside}

% Version online
\geometry{a4paper, vmargin=2.5cm, hmargin=2.9cm, top=1.3in, headheight=15pt, twoside}


% Chargement des couleurs de la charte graphique de l'IPP %
\definecolor{ippdark}{RGB}{0,81,101}
\definecolor{ipplight}{RGB}{0,138,155}

% Pour s�lectionner la police "sans serif" de la classe "Charter" %
\fontfamily{sffamily}\fontseries{m}\selectfont

% Style des chapitres, sections, etc. %
\chapterfont{\color{ipplight}{}\fontfamily{pag}\fontseries{b}\fontshape{sc}\selectfont}
\sectionfont{\color{ippdark}{}\fontfamily{pag}\fontseries{b}\fontshape{n}\selectfont}
\subsectionfont{\color{ippdark}{}\fontfamily{pag}\fontseries{b}\fontshape{n}\selectfont}
\subsubsectionfont{\color{ippdark}{}\fontfamily{pag}\fontseries{b}\fontshape{n}\selectfont}

% Style des en-t�te %
\newcommand{\enteteone}[2][]{\pagestyle{fancy}\fancyhead{#1} \lhead[\textsl{#2}]{} \rhead[]{\textsl{#2}}}
\newcommand{\entetetwo}[3][]{\pagestyle{fancy}\fancyhead{#1} \lhead[\textsl{#2}]{} \rhead[]{\textsl{#3}}}

% Style des pied-de-page %
%\newcommand{\footertwo}{\pagestyle{fancy}\fancyfoot{} \lfoot[Institut des politiques publiques]{\thepage} \rfoot[\thepage]{Institut des politiques publiques}}


% Modification du libell� de la table des mati�res et des listes de tableaux et figures %
\addto\captionsfrench{%
  \renewcommand{\tablename}{\bf \color{ippdark} \textsc{Tableau}}%
  \renewcommand{\figurename}{\bf \color{ippdark} \textsc{Figure}}%
  \renewcommand{\contentsname}{Sommaire}%
  \renewcommand{\listfigurename}{Liste des figures}%
  \renewcommand{\listtablename}{Liste des tableaux}%
}

% Pour �viter le message de warning sur l'inclusion de certains fichiers pdf
\pdfoptionpdfinclusionerrorlevel=0
\pdfoptionpdfminorversion=6
