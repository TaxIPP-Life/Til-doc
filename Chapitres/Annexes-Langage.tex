Pour un mod�le de micro-simulation, la question du langage de programmation retenu est cruciale mais non trivial. 

Le lecteur int�ress� est renvoy� au document de Didier Blanchet. Ce document milite pour l'utilisation de R. 
R a effectivement d'ind�niables atouts qui en font probablement le meilleur choix pour faire de la microsimulation avec un logiciel de statistiques. Toutefois, la microsimulation n'est pas de la statistique classique. De plus, la question de la performance est primordiale pour un projet comme TaxIpp-Life qui veut g�n�rer une population sur longue p�riode, �ventuellement, mois par mois, appliquer la l�gislation socio-fiscale fran�aise et le tout sur un �chantillon potentiellement grand (le million d'individu ne doit pas �tre dirimant).
Ces raisons nous ont amen� vers l'utilisation de Python, un langage assez simple d'utilisation pour �tre pris en main par des non-sp�cialistes de la programmation est assez performant pour justifier son apprentissage par rapport � R. 
A l'usage, on se rend aussi compte que pour le cas pr�cis de la microsimulation la syntaxe de Python, son syst�me de classe imbriqu�e les unes dans les autres est un vrai atout, oserai-je dire un vrai plaisir. 
Ce choix a �t� aussi motiv� par un contexte important sur lequel nous reviendrons. Le projet Liam2 ainsi que le projet OpenFisca qui ont tous les deux, et ind�pendamment, on fait le choix, r�cemment, d'utiliser Python. TaxIpp-Life s'appuie sur ces deux projets et leur doit beaucoup. 
Pour autant, R n'a pas �t� mis au placard. En utilisant au possible les transferts de donn�es et m�me l'imbrication de plus en plus possible de code R et de code Python, on tire au maximum profit de ces deux langages. Le traitement initial des donn�es et l'exploitation finale sont faites en R, tandis que la partie microsimulation pure est r�alis�e sous Python. 
